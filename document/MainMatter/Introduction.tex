\chapter*{Introducción}\label{chapter:introduction}
\addcontentsline{toc}{chapter}{Introducción}

Hyperledger Fabric es una plataforma blockchain respaldada por una arquitectura modular que ofrece un alta grado de confidencialidad, resiliencia, flexibilidad y escalabilidad. Est\'a dise\~nada para admitir implementaciones de diferentes componentes y adaptarse a las complejidades que existen en el ecosistema empresarial.\\

Propone un nuevo enfoque en su arquitectura para transacciones, conocido como $ejecutar-ordenar-validar$. Ejecutar una transacci\'on implica, primero, comprobar su correcci\'on, luego en el proceso de ordenaci\'on se efect\'ua un protocolo de consenso, y finalmente se chequea la validez de la transacci\'on contra una pol\'itica de endoso espec\'ifica de la aplicaci\'on antes de confirmarla en el ledger. Su r\'apida expansi\'on en el mercado ha incentivado a la comunidad de desarrolladores a contribuir a su constante evoluci\'on. Actualmente ocupa diversos casos de uso de relevancia, como $Global Trade Digitization$ [33], $SecureKey$ [15], $Everledger$ [5]. 



Debido a los numerosos componentes y fases, Fabric proporciona varios par\'ametros configurables, como el tama\'~no del bloque, la pol\'itica de respaldo, los canales, base de datos de estado. De ah\'i que uno de los principales retos a la hora de establecer una red blockchain eficiente es encontrar el conjunto correcto de valores para estos par\'ametros.\\

Motivaci\'on:\\

Con los [\cite{watson53}] nuevos cambios introducidos en Hyperledger Fabric hace m\'as demandada la plataforma en el sector empresarial, producto a su adaptabilidad en escenarios de consensos y pol\'iticas de gobierno para varias entidades. Resulta esencial el rendimiento de la plataforma en la transacci\'on de informaci\'on, lo cual nos motiva a realizar un estudio que posibilite obtener un conjunto de configuraciones \'optimas.
