\chapter{Estado del Arte de la plataforma blockchain Hyperledger Fabric}\label{chapter:state-of-the-art}

Hyperledger Fabric es una de las plataformas blockchain m\'as populares, administrada por Linux Foundation. Se basa en una plataforma privada donde solo los usuarios autenticados participan en ella. Difiere de las plataformas blockchain p\'ublicas que posibilitan la uni\'on de cualquier usuario a la red. Adem\'as, Fabric presenta una arquitectura de ejecuci\'on, orden y validaci\'on que supera los l\'imites de la arquitectura de orden y ejecuci\'on anterior []. Esto mejora sustancialmente la escalabilidad de rendimiento en redes blockchain con un n\'umero elevado de Peers, lo que permite a Fabric ser competente en Global Trade Digitalization [], SecureKey [] y Everledger [].Constituye la primera plataforma blockchain que admite contratos inteligentes creados en lenguajes de programaci\'on de uso general como Java, Golang y Node.js; siendo factible para la mayor\'ia de las empresas en el desarrollo de los contratos inteligentes, sin necesidad de capacidad adicional para aprender lenguajes espec\'ificos de dominio restringidos, conocidos por sus siglas DSL. 

Fabric es compatible con protocolos de consenso conectables que permiten a la plataforma personalizarse de manera eficaz de acuerdo al caso de uso y modelos de confianza de los entes que la integran. 
