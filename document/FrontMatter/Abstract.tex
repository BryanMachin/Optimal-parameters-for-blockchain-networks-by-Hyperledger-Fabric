\begin{resumen}
Hyperledger Fabric es actualmente una de las plataformas \emph{Blockchain} comerciales m\'as populares. Con la posibilidad de ejecutar contratos inteligentes desarrollados en lenguajes de programaci\'on de prop\'osito general, Fabric se ha convertido en una de las plataformas m\'as populares en el \'area empresarial. Entre sus bondades tambi\'en se tiene un conjunto de par\'ametros configurables, que en dependencia del volumen de transacciones por segundo a la que opere la red, pueden mejorar o no, su rendimiento. Para un desarrollador potencialmente interesado, no es trivial decidir si una determinada configuraci\'on de red de Fabric cumplir\'a con las expectativas requeridas con respecto al rendimiento. Por lo tanto, este trabajo muestra un an\'alisis de rendimiento para Hyperledger Fabric v2.4 y se centra en ofrecer una configuraci\'on para dos posibles escenarios de trabajo de: bajos y altos vol\'umenes de transacciones por segundo. Las m\'etricas de latencia, junto con la escalabilidad en los nodos de la plataforma y el tama\~no en los bloques del canal de comunicaci\'on son las variables de estudio analizadas. Los resultados muestran, de acuerdo a los par\'ametros ofrecidos como candidatos, para cada uno de los escenarios planteados, una configuraci\'on que optimiza el rendimiento o minimiza la latencia media, en dependencia de las necesidades del operador de la red.
\end{resumen}

\begin{abstract}
Hyperledger Fabric is currently one of the most popular commercial Blockchain platforms. With the ability to run smart contracts developed in general-purpose programming languages, Fabric has become one of the most popular platforms in the enterprise arena. Among its benefits there is also a set of configurable parameters, which depending on the volume of transactions per second at which the network operates, may or may not improve its performance. For a potentially interested developer, it is not trivial to decide whether a given fabric configuration will meet the required performance expectations. Therefore, this work shows a performance analysis for Hyperledger Fabric v2.4 and focuses on offering a configuration for two possible work scenarios: low and high volumes of transactions per second. The latency metrics, along with the scalability in the platform nodes and the size of the communication channel blocks are the study variables analyzed. The results show, according to the parameters offered as candidates, for each of the proposed scenarios, a configuration that optimizes performance or minimizes average latency, depending on the needs of the network operator.
\end{abstract}