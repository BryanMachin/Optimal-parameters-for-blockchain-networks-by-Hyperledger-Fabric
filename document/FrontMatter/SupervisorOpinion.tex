\begin{opinion}
El trabajo de diploma \emph{An\'alisis de rendimiento en Hyperledger Fabric} del estudiante Bryan Mach\'in Garc\'ia para optar por el t\'itulo de Licenciado en Ciencia de la Computaci\'on, es un tema de investigaci\'on de suma importancia para el Instituto de Criptograf\'ia porque permite mejorar la eficiencia de los servicios basados en redes \emph{blockchain} de Hyperledger Fabric.\\

El diplomante ha mostrado inter\'es por la investigaci\'on, ha sido receptivo a las sugerencias, cr\'iticas y opiniones de ambos tutores; ganando en conocimiento, dominio del tema, y habilidades para reajustar sus experimentos y presentar resultados.\\
 
Puedo afirmar que ha mostrado disciplina y dedicaci\'on en la realizaci\'on de las tareas, tanto en la redacci\'on del trabajo de diploma, como en la organizaci\'on e investigaci\'on del tema, lo cual se ve reflejado en los resultados entregados por el diplomante. Para ello, comenz\'o con la asimilaci\'on y estudio de las tecnolog\'ias indicadas por los tutores, mostrando adem\'as buenas capacidades de asimilaci\'on e independencia.\\

En consecuencia, se define que la tesis cumple con el rigor metodol\'ogico, cient\'ifico y est\'a en funci\'on de los requisitos definidos, partiendo adem\'as del estudio de fuentes y publicaciones recientes relacionadas al tema de investigaci\'on.\\

Por tanto, hago constar que la tesis re\'une los est\'andares metodol\'ogicos exigidos por la Facultad de Matem\'atica y Computaci\'on de la Universidad de la Habana, para ser presentada y sometida a evaluaci\'on en su ejercicio de defensa.\\

Felicito al diplomante por haber respondido con responsabilidad al desaf\'io del estudio y haber finalizado exitosamente su trabajo de diploma.\\

MsC. Camilo Denis González
\end{opinion}