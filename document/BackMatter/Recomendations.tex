\begin{recomendations}
El estudio se realiz\'o sobre una muestra de par\'ametros de configuraci\'on, y se determin\'o con respecto a ellos, los que optimizan escenarios de 10 y 100 transacciones por segundo. Propongo extender la investigaci\'on para una muestra de tama\~no mayor, que permita realizar un an\'alisis en pruebas de hip\'otesis para tratar de obtener una generaliaci\'on en el comportamiento de la variaci\'on de los par\'ametros.\\

Determinar un tama\~no adecuado de los bloques en un canal de comunicaci\'on, resulta una tarea compleja producto a la influencia directa que tiene sobre varias de las componentes de fabric, como los nodos pares y los nodos ordenadores. Por lo que trabajar en una herramienta que itere de manera eficiente para converger hacia un posible candidato a tama\~no \'optimo ayudar\'ia en gran medida a encontrar los valores que mejoren el rendimiento en un escenario determinado.
\end{recomendations}
