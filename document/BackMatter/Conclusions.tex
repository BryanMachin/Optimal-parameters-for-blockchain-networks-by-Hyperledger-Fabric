\begin{conclusions}
Los resultados obtenidos en el desarrollo de la investigaci\'on posibilitan un desempe\~no, en la plataforma de Hyperledger Fabric, m\'as eficiente para los escenarios planteados. Si se desea operar una red con un bajo volumen de transacciones por segundo, se ofrecen configuraciones que garantizan para 10 TPS un elevado valor de rendimiento o una baja latencia media de la red, de acuerdo a los requerimientos particulares del operador de red. Para optimizar el rendimiento las tres configuraciones de tama\~no de bloque para canales de comunicaci\'on: 10, 50 y 100; ofrecen un elevado por ciento de eficiencia, siendo estos mayores o iguales al 96$\%$. En el caso de la latencia para canales de tama\~no 50 y 100 sus valores son suficientemente homog\'eneos como para decantarnos por cualquiera de ellos, pero son superiores en aproximadamente un 25$\%$, al valor de latencia para redes con canales de tama\~no 10. Siendo la configuraci\'on, por defecto, de Hyperledger Fabric con valor 10, una excelente apuesta, para este tipo de escenarios de bajos vol\'umenes de transacciones por segundo.\\

Para escenarios donde el volumen de transacciones por segundo sea elevado y a su vez pr\'oximos a las 100 TPS, se ofrece a 100 como candidato a \'optimo local de acuerdo al estudio de los tama\~nos de bloque de: 10, 50 y 100. Este valor garantiza un promedio de rendimiento superior al 80$\%$ y un valor en la latencia por debajo de los 11 segundos. La configuraci\'on, por defecto, de Fabric para el tama\~no de los bloques en los canales, en estos escenarios de altos valores de TPS ocasiona muy bajo rendimiento, el cual no supera el 10$\%$ de TPS y un valor de latencia superior a los 90 segundos. Por tanto, confirma la necesidad del estudio de par\'ametros \'optimos para determinados casos de uso, no siendo siempre la configuraci\'on establecida por el proveedor de la plataforma una opci\'on a tener en cuenta.  
\end{conclusions}
